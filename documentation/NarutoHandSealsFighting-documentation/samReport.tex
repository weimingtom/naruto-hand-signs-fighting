\documentclass[a4paper,10pt, twocolumn]{article}

\usepackage[utf8x]{inputenc}
\usepackage{amssymb,amsmath}
\usepackage{graphicx}
\usepackage[italian]{babel}
\usepackage{hyperref}

\title{Relazione al progetto: \\ Naruto Hand Seals Fighting\\ del corso di\\Sistemi e Applicazioni
Multimediali}
\author{Michele Tamburini \\ mtamburi@cs.unibo.it}
\makeindex

\begin{document}

\maketitle

\section{Abstract}
Il progetto nasce con lo scopo di esplorare uno dei campi della \emph{Augmented Reality}, che
prevede l'uso delle mani nell'interazione uomo-macchina.
La realizzazione ultima consiste di un gioco di combattimento in modalità "picchiaduro"
che prende ambientazione e scenografia dall'ormai noto anime giapponese "Naruto".
Le mosse per così dire speciali, dovranno essere attivate non con semplici
combinazioni di tasti, ma attraverso le posizioni ed i gesti delle mani dell'utente,
acquisiti attraverso una webcam, che saranno poi analizzati da un motore di
riconoscimento per verificarne l'accuratezza.
L'implementazione del presente prevede il 
motore di riconoscimento e una semplice sezione di addestramento
per l'utente, che lo coinvolga nell'apprendimento di alcune mosse.\\
Vengono utilizzate librerie \emph{OpenCV} per l'analisi delle immagini, 
\emph{Guichan} per l'interfaccia grafica, \emph{SDL} ed il linguaggio C++.

\section{Stato dell'arte}
La ricerca di tecniche che favoriscano metodologie di utilizzo dei calcolatori
attraverso forme di interazione sempre pi\'{u} innovative, quali ad esempio
l'uso del proprio corpo, ha aperto la strada gi\'{a} da qualche tempo
a nuove sfide per il mondo informatico.\\
Il campo di ricerca dell'\emph{``hand tracking''} vuole dare all'utilizzatore
del calcolatore la possibilit\'{a} di un controllo, completo o parziale, di questo,
attraverso i complicati movimenti o stimoli che la mano umana \'{e} in grado di fornire.
L'acquisione di questi, sia essa agevolata dall'ausiglio di sensori particolari,
guanti o addirittura con la mano nuda, risulta essere la prima problematica da affrontare,
comune a tutte le proposte fino ad ora osservate (\cite{handTrackHCI} \cite{handTrackNoMarkers}
 \cite{handNavigator} \cite{hugeSurvey} \cite{mitGlove}. 
 Gli spunti proposti nella bibliografia, lontano
dall'essere esaustiva sull'argomento, vogliono semplicemente offrire una panoramica ad ampio
spettro di alcune soluzioni a tale problema.

\section{Progetto: introduzione}
  \subsection{Ambientazione}
  L'ormai celeberrimo anime ``Naruto'' ha portato anche in Italia una storia di fanstasia
  dalle forti connotazioni orientali. Tra gli ingredienti maggiormente apprezzati dal
  grande pubblico risiede il fatto che i personaggi utilizzano movimenti 
  delle mani estremamente complicati per lanciare mosse di arti marziali speciali.
  Ognuna di queste \`{e} composta da un numero variabile di ``sigilli'' (posizioni degli arti
  superiori) che corrispondo ai 12 segni zodiacali cinesi.\\
  Nel seguente progetto ho inserito tutti e soli tali gesti, ovvero non si tengono in considerazione
  movimenti diversi (come il battito delle mani) che pur compaiono nell'anime.
  Lo scopo del giocatore sar\`{a} quello di riprodurre
  alcune posisioni che vedr\`{a} raffigurate sottoforma di immagini. Ciascuna
  di queste \`{e} denominata segno, o sigillo (signs, seals). Una mossa (Move) \`{e} dunque
  costituita da un numero variabile di segni presi tra i 12 sopra descritti.
  
  \subsection{Il nostro problema nel dettaglio}
  In questo lavoro non ho certo la pretesa di affrontare un tema cos\'{i} imponente come 
  quello dell'hand tracking. Il problema si riduce infatti ad una analisi di immagini
  seppur non priva di compromessi e difficolt\`{a}. A differenza di alcuni lavori
  di spicco che usano un database e oggetti esterni
  per l'inferenza della reale posizione della mano \cite{mitGlove}, in questo caso non
  \`{e} necessario avere una precisa cognizione del punto di partenza e di 
  fine dell'arto.
  Il giocatore \`{e} libero di eseguire qualunque movimento e, una volta pronto,
  aziona l'acquisione del gesto, che si traduce per il calcolatore con: l'acquisizione 
  di acluni frame di immagine, trasformazione di questi con tecniche di processazione
  dell'immagine e confronto con un template precedentemente preparato.\\
  Un punto che abbatte notevolmente la dimensione del problema, ed evita quindi l'uso 
  di un database, si basa proprio su una carattestica derivata dall'ambientazione:
  sebbene le combinazioni che possono avere luogo siano infinite (permutazioni di 12 
  elementi di lunghezza arbitraria) i diversi gesti da riconoscere rimangono comunque 12.
  
  \subsection{Pensando in grande}
  Fondamentalmente questo gioco risiede nella categoria ``picchiaduro''.
  L'ingrediente della \emph{Augmented Reality} emerge nel
  momento in cui l'utente abbandona la tastiera per interagire con le proprie mani 
  attraverso la webcam ed eseguire dal vivo i gesti necessari per la mossa selezionata.\\
  Per ciascuno gli verr\`{a} notificato il livello di accuratezza 
  calcolato in una scala da 0 a 100.
  Al termine dell'ultimo viene poi restituito il voto medio.
  
%  \subsection{Goal Oriented Design}
%  In questa sezione introduco in maniera assolutamente superficiale ed approssimata 
%  alcune linee guida per il personaggio principale e quello secondario pi\`{u} rilevante.\\
%  Tale citazione ha solo lo scopo di fornire una migliore rappresentazione di uno scenario 
%  di utilizzo dell'applicazione.
%  \begin{itemize}
%   \item[Principale] Lela, donna sulla quarantina, sposata e madre di Gioacchino. 
%   Ha un impiego presso un'impresa 
%  \end{itemize}


  
  
\section{Descrizione del progetto}
  \subsection{Sezione di Acquisizione}
  \subsection{Logica di Gioco}
  \subsection{Stack di Gioco}
  \subsection{L'interfaccia Grafica}
  \subsection{La macchina per il Riconoscimento}
  
\section{Riconoscimento}
  \subsection{``The Recognition Engine''}
  \subsection{I moduli}
  \subsection{Le strategie}
 
\section{Alcuni Test}

\section{Osservazioni}

\section{Coclusioni}


\clearpage

\nocite{*}

\bibliographystyle{plain}
\bibliography{samBibliography}

\end{document}